\chapter{Introdução}
\label{cap:introducao}

Jogos em geral, sempre tiveram um papel importante socialmente ao servir, muitas vezes como fio condutor para interações social entre familiares, amigos e até mesmo desconhecidos que por muitas vezes acabam por desenvolver laços mais tarde. Além disso, Jogos também tem mostrado ao longo do tempo seu valor competitivo, ou seja, podem ser usados para guiar competições como o CBLOL que movimentam um ecossistema econômico de torcedores, jogadores e equipes, onde muitos podem até mesmo viver exclusivamente de jogar e é capaz de lotar arenas \cite{FinalCBL40:online}. Além disso, jogos podem ser benéficos para desenvolvimento e manutenção de habilidades cognitivas como leitura \cite{Enhancingreading} ou memoria \cite{videogameimproveshigh-fidelitymemory}.

Contudo, pessoas com deficiência física muitas vezes acabam por ser excluídas desses jogos e de poderem aproveitar seus benefícios, e dentre esses grupos as pessoas com deficiência visual grave ou completamente cegas acabam por ser um dos grupos mais prejudicados e excluídos em relação aos benefícios dos jogos.

Nesse contexto, surge a necessidade do desenvolvimento de soluções e jogos capazes não so de atender as pessoas portadores de tais deficiências mas de integra-las a outras pessoas, uma vez que a capacidade integradora dos jogos é umas das perspectivas mais importantes e produtivas socialmente. 

Dessa forma, buscando alcançar esse objetivo, esse projeto propõe como o um jogo, neste caso um jogo da memoria, pode ser desenvolvido fazendo uso das tecnologias existentes no mercado afim de produzir um experiência integradora entre pessoas que possuem diferentes deficiências visuais e pessoas que não as possuem. 

\section{Perguntas de partida}

Assim, a partir dessa proposta proposta surgem as seguintes perguntas de partida, com relação ao desenvolvimento específico da solução:

\begin{questao}
  \item Quais estratégias da engenharia de software podem ser empregadas no desenvolvimento de jogos online multiplayer via navegador para garantir a qualidade do software?
  \item É possível realizar, dentro da dinâmica de um jogo em tempo real via navegador, uma experiência de uso responsiva e veloz?
  \item Existem tecnologias dentro do software livre podem conseguir fazer um jogo online em tempo real ser acessível para pessoal cegas ou com deficiência visual grave?
\end{questao}

\section{Objetivos Gerais}

O objetivo desse projeto é construir uma tecnologia que permita a criação, manutenção e distribuição de desafios de jogo da memória online multiplayer com acessibilidade total para pessoas cegas ou com deficiência visual grave. Nesse sentido, superar por meio da tecnologia as barreiras que a deficiência impõem ao integrar pessoas com deficiência com pessoas sem deficiência dentro de uma atividade recreativa de jogos online e o principal foco desse desenvolvimento.

Dessa forma, um bom uso das técnicas de desenvolvimento web, em conjunto de técnicas técnicas de engenharia de software para planejar, projetar e desenvolver o software, assim como, a construção de estruturas e nuvem serão conhecimentos essenciais para superar esse desafio.

Além disso, é importante que este desafio seja superado de forma consistente, durável e estendível. Ou seja, o software deve ser construído de modo a facilitar manutenções, adições de novas funções e bom desempenho ao longo do tempo alcançando os objetivos descritos pela engenharia de software

\section{Objetivos do software como produto}

Nessa perspectiva, surgem como objetivos finais, construir um software capaz de alcançar de maneira razoável os atributos definidos pela ISO/IEC, ou seja: \cite{ISO/IEC9126}

\begin{alineascomponto}
  \item Funcionalidade
  \item Confiabilidade
  \item Usabilidade
  \item Eficiência
  \item Manutenibilidade
  \item Portabilidade
\end{alineascomponto}


