\chapter{Perguntas de partida}

\label{chap:Perguntas de partida}


Jogos em geral, sempre tiveram um papel importante socialmente ao servir, muitas vezes como  fio condutor para interações social entre familiares amigos e, até mesmo desconhecidos que por muitas vezes acabam por desenvolver laços mais tarde. Além disso, Jogos também tem mostrada ao longo do tempo seu valor competitivo, ou seja, podem ser usados para guiar competições como a de xadrez que movimentam um ecossistema de pessoas em volta, onde muito podem até mesmo viver exclusivamente de jogar algum jogo, ou de propiciar condições para que o jogos aconteçam sempre com plateia pagante. Por fim fica claro que os jogos em geral ja possuem um papel importante socialmente seja como ferramenta de socialização e integração ou como atividade econômica através de competições ou venda.


Contudo, pessoas com deficiência física muitas vezes acabam por ser excluídas desses jogos e de poderem aproveitar seus benefícios e dentre esses grupo as pessoas com deficiência visual grave ou completamente cegas acabam por ser um dos grupos mas prejudicados e excluídos em relação ao papel dos jogos.

Nesse contexto, surge a necessidade do desenvolvimento de soluções e jogos capazes não so de atender as pessoas portadores de tal deficiência mas integra-las a outras pessoas, uma vez que a capacidade integradora dos jogos é umas das perspectivas mais importantes e produtivas socialmente. Dessa forma, buscando alcançar esse objetivo, esse projeto propõe como o um jogo, neste caso um jogo da memoria, pode ser desenvolvido fazendo uso das tecnologias existentes no mercado afim de produzir um experiência integradora entre pessoas que possuem diferentes deficiências visuais e pessoas que não as possuem. 

Assim, dessa proposta surgem as seguintes perguntas de partida:

\begin{questao}
  \item Quais estratégias da engenharia de software podem ser empregadas no desenvolvimento de jogos online multiplayer que funcionam via navegador para garantir a qualidade do software?
  \item Como garantir, dentro da dinâmica de um jogo em tempo real via navegador, uma experiência de uso responsiva e veloz?
  \item Exitem tecnologias dentro do software livre podem conseguir fazer um jogo online em tempo real ser acessível para pessoal cegas ou com deficiência visual grave?
\end{questao}
