% Para utilizar este template siga o tutorial disponível em http://www.biblioteca.ufc.br/wp-content/uploads/2015/09/tutorial-sharelatex.pdf

%%%%%%%%%%%%%%%%%%%%%%%%%%%%%%%%%%%%%%%%%%%%%%%%%%%%%%%
%% Você deve criar uma conta no Overleaf. Depois,    %%
%% vá nas opções no canto esquerdo superior da tela  %%
%% e clique em "Copiar Projeto". Dê um novo nome pa- %%
%% ra o projeto.                                     %%
%%                                                   %%
%% Os principais desenvolvedores deste template são: %%
%%                                                   %%
%%            Ednardo Moreira Rodrigues              %%
%%       (Doutor em Engenharia Elétrica - UFC)       %%
%%                      &                            %%
%%            Alan Batista de Oliveira               %%
%%           (Engenheiro Eletricista - UFC)          %%
%%                                                   %%
%% Consultoria Bibliotecária                         %%
%%                                                   %%
%%  Versão 2016 - ShareLaTeX:                        %% 
%%                                                   %%
%% - Francisco Edvander Pires Santos;                %%
%% - Juliana Soares Lima;                            %%
%% - Izabel Lima dos Santos;                         %%
%% - Kalline Yasmin Soares Feitosa;                  %%
%% - Eliene Maria Vieira de Moura.                   %%
%%
%%  Versão 2019 - Overleaf:
%%  
%%  Biblioteca de Ciências Humanas: 
%% - Francisco Edvander Pires Santos;                %%
%% - Juliana Soares Lima;                            %%
%% - Eliene Maria Vieira de Moura;                   %%
%% - Edmundo Moreira de Sousa Filho.                 %%
%%                                                   %%
%% Biblioteca da FEAAC:                              %%
%% - Izabel Lima dos Santos;                         %%
%% - Kalline Yasmin Soares Feitosa;                  %%
%% - Kleber Lima dos Santos.                         %%
%%                                                   %%
%%  Biblioteca do Curso de Física:                   %%
%% - Aline Rodrigues de Lima Mendes;                 %%
%% - Maria de Jesus Silva dos Santos.                %%
%%                                                   %%
%%  Biblioteca Central do Campus do Pici:            %%
%% - Raquel da Silva Nascimento.                     %%
%%                                                   %%
%% Colaboradores                                     %%
%%                                                   %%
%% -Andrei Bosco Bezerra Torres                      %% 
%% (Professor - Sistemas e Mídias Digitais -         %%
%% Instituto Universidade Virtual - UFC)             %%
%% Tiago ALves Lima                                  %% 
%% (Aluno de Mestrado em Eng. Elétrica)              %%
%%                                                   %%
%% Adaptação para o Inglês                           %%
%% Matheus Gomes Correia                             %%
%% (Aluno de Mestrado em Eng. de Tranportes - UFC    %%
%%                                                   %%
%% Grande parte do trabalho foi adaptado do template %%
%% da UECE elaborado por:                            %%
%% Thiago Nascimento  (UECE)                         %%
%% Project available on:                             %%
%% https://github.com/thiagodnf/uecetex2             %%
%%                                                   %%
%% "Dúvidas, esclarecimentos ou sugestões podem ser  %%
%% enviadas para o seguinte e-mail:                  %%
%%                                                   %%
%%             atendimentobch@ufc.br                 %%
%%                                                   %%
%% As últimas atualizações estão descritas no inicio %%
%% do arquivo "README.md".                           %%
%%                                                   %%
%%%%%%%%%%%%%%%%%%%%%%%%%%%%%%%%%%%%%%%%%%%%%%%%%%%%%%%

\documentclass[        
    a4paper,          % Tamanho da folha A4
    12pt,             % Tamanho da fonte 12pt
    chapter=TITLE,    % Todos os capitulos devem ter caixa alta
    section=Title,    % Todas as secoes devem ter caixa alta somente na primeira letra
    subsection=Title, % Todas as subsecoes devem ter caixa alta somente na primeira letra
    oneside,          % Usada para impressao em apenas uma face do papel
    english,          % Hifenizacoes em ingles
    spanish,          % Hifenizacoes em espanhol
    brazil,           % Ultimo idioma eh o idioma padrao do documento
    fleqn             % Comente esta linha se quiser centralizar as equacoes. Comente também a linha 65 abaixo
]{abntex2}

\input{lib/preambulo}

\setlength{\mathindent}{0pt} %Complementa o alinhamento de equações para totalmente a esquerda.

%%%%%%%%%%%%%%%%%%%%%%%%%%%%%%%%%%%%%%%%%%%%%%%%%%%%%
%%                     ATENCAO                     %%
%%%%%%%%%%%%%%%%%%%%%%%%%%%%%%%%%%%%%%%%%%%%%%%%%%%%%
%  Qual e o nivel do trabalho academico que voce esta 
% escrevendo? Retire o simbolo "%" apenas de um dos 
% quatro topicos abaixo refente ao nível do seu traba
% -lho.

%\trabalhoacademico{tccgraduacao}
%\trabalhoacademico{tccespecializacao}
%\trabalhoacademico{dissertacao}
\trabalhoacademico{tccgraduacao}

%%%%%%%%%%%%%%%%%%%%%%%%%%%%%%%%%%%%%%%%%%%%%%%%%%%%%

% Define se o trabalho e uma qualificacao
% Coloque 'nao' para versao final do trabalho

\ehqualificacao{nao}

% Remove as bordas vermelhas e verdes do PDF gerado
% Coloque 'sim' pare remover

\removerbordasdohyperlink{sim} 

% Adiciona a cor Azul a todos os hyperlinks

\cordohyperlink{nao}

%%%%%%%%%%%%%%%%%%%%%%%%%%%%%%%%%%%%%%%%%%%%%%%%%%%%%
%%         Informacao sobre a instituicao          %%
%%%%%%%%%%%%%%%%%%%%%%%%%%%%%%%%%%%%%%%%%%%%%%%%%%%%%

\ies{Universidade Federal do Ceará}
\iessigla{UFC}
\centro{Centro de Tecnologia}
\departamento{Departamento de Teleinfomática(DETI)}

%%%%%%%%%%%%%%%%%%%%%%%%%%%%%%%%%%%%%%%%%%%%%%%%%%%%%
%%        Informacao para TCC de Graduacao         %%
%%%%%%%%%%%%%%%%%%%%%%%%%%%%%%%%%%%%%%%%%%%%%%%%%%%%%

\graduacaoem{Engenharia de Computação}
\habilitacao{Bachelor} % Ou 'Licentiate'

% AVISO: Caso necessario alterar o texto de apresenta-
% cao da Especializacao, ir a pasta "lib", arquivo 
% "ufctex.sty" na linha 512.


%%%%%%%%%%%%%%%%%%%%%%%%%%%%%%%%%%%%%%%%%%%%%%%%%%%%%
%%     Informacao para TCC de Especializacao       %%
%%%%%%%%%%%%%%%%%%%%%%%%%%%%%%%%%%%%%%%%%%%%%%%%%%%%%

\especializacaoem{Yyyyyyyyy}

% AVISO: Caso necessario alterar o texto de apresenta-
% cao da Especializacao, ir a pasta "lib", arquivo 
% "ufctex.sty" na linha 517.

%%%%%%%%%%%%%%%%%%%%%%%%%%%%%%%%%%%%%%%%%%%%%%%%%%%%%
%%         Informacao para Dissertacao             %%
%%%%%%%%%%%%%%%%%%%%%%%%%%%%%%%%%%%%%%%%%%%%%%%%%%%%%

\programamestrado{Post-Graduation Program in Xxxxxxx}
\nomedomestrado{Master Degree in Xxxxxxx}
\mestreem{xxxxxx Engineering}
\areadeconcentracaomestrado{xxxxxx Engineering}

% AVISO: Caso necessario alterar o texto de apresenta-
% cao da dissertacao, ir a pasta "lib", arquivo 
% "ufctex.sty" na linha 521.

%%%%%%%%%%%%%%%%%%%%%%%%%%%%%%%%%%%%%%%%%%%%%%%%%%%%%
%%               Informação para Tese              %%
%%%%%%%%%%%%%%%%%%%%%%%%%%%%%%%%%%%%%%%%%%%%%%%%%%%%%

\programadoutorado{Post-Graduation Program in Xxxxxx}
\nomedodoutorado{Doctoral Degree in Xxxxxxx}
\doutorem{xxxxxx Engineering}
\areadeconcentracaodoutorado{xxxxxx Engineering}

% AVISO: Caso necessario alterar o texto de apresenta-
% cao da tese, ir a pasta "lib", arquivo "ufctex.sty" 
% na linha 525.

%%%%%%%%%%%%%%%%%%%%%%%%%%%%%%%%%%%%%%%%%%%%%%%%%%%%%
%%      Informacoes relacionadas ao trabalho       %%
%%%%%%%%%%%%%%%%%%%%%%%%%%%%%%%%%%%%%%%%%%%%%%%%%%%%%

\autor{Luis Gustavo de Castro Sousa}
\titulo{Desenvolvimento de um jogo da memoria online com acessibilidade para pessoas com deficiência visual.}
\data{2022}
\local{Fortaleza}

% Exemplo: \dataaprovacao{6th September 2019}
\dataaprovacao{}

%%%%%%%%%%%%%%%%%%%%%%%%%%%%%%%%%%%%%%%%%%%%%%%%%%%%%
%%           Informação sobre o Orientador         %%
%%%%%%%%%%%%%%%%%%%%%%%%%%%%%%%%%%%%%%%%%%%%%%%%%%%%%

\orientador{Prof. Dr. Xxxxxxx Xxxxxx Xxxxxxx}
\orientadories{Federal University of Ceará (UFC)}
\orientadorcentro{Centro de Ciências e Tecnologia (CCT)}
\orientadorfeminino{nao} % Coloque 'sim' se for do sexo feminino

%%%%%%%%%%%%%%%%%%%%%%%%%%%%%%%%%%%%%%%%%%%%%%%%%%%%%
%%          Informação sobre o Coorientador        %%
%%%%%%%%%%%%%%%%%%%%%%%%%%%%%%%%%%%%%%%%%%%%%%%%%%%%%

% Deixe o nome do coorientador em branco para remover do documento

\coorientador{}
\coorientadories{Universidade Coorientador (SIGLA)}
\coorientadorcentro{Centro do Coorientador (SIGLA)}
\coorientadorfeminino{nao} % Coloque 'sim' se for do sexo feminino

%%%%%%%%%%%%%%%%%%%%%%%%%%%%%%%%%%%%%%%%%%%%%%%%%%%%%
%%              Informação sobre a banca           %%
%%%%%%%%%%%%%%%%%%%%%%%%%%%%%%%%%%%%%%%%%%%%%%%%%%%%%

% Atenção! Deixe em branco o nome do membro da banca para remover da folha de aprovacao

% Exemplo de uso:
% \membrodabancadois{Prof. Dr. Fulano de Tal}
% \membrodabancadoisies{Universidade Federal do Ceará - UFC}


\membrodabancadois{Prof. Dr. Xxxxxxx Xxxxxx Xxxxxxx}
\membrodabancadoiscentro{Faculdade de Filosofia Dom Aureliano Matos (FAFIDAM)}
\membrodabancadoisies{Universidade do Membro da Banca Dois (SIGLA)}
\membrodabancatres{Prof. Dr. Xxxxxxx Xxxxxx Xxxxxxx}
\membrodabancatrescentro{Centro de Ciências e Tecnologia (CCT)}
\membrodabancatresies{Universidade do Membro da Banca Três (SIGLA)}
\membrodabancaquatro{Prof. Dr. Xxxxxxx Xxxxxx Xxxxxxx}
\membrodabancaquatrocentro{Centro de Ciências e Tecnologia (CCT)}
\membrodabancaquatroies{Universidade do Membro da Banca Quatro (SIGLA)}
\membrodabancacinco{Prof. Dr. Xxxxxxx Xxxxxx Xxxxxxx}
\membrodabancacincocentro{Teste}
\membrodabancacincoies{Universidade do Membro da Banca Cinco (SIGLA)}
\membrodabancaseis{Prof. Dr. Xxxxxxx Xxxxxx Xxxxxxx}
\membrodabancaseiscentro{}
\membrodabancaseisies{Universidade do Membro da Banca Seis (SIGLA)}

\begin{document}	

    % \selectlanguage{english} % Colocando em inglês
    \bibliographystyle{lib/abntex2-alf} % Ajustes das referências em inglês
    
	% Elementos pré-textuais
	\imprimircapa
	\imprimirfolhaderosto{}
	%\imprimirfichacatalografica{1-pre-textuais/ficha-catalografica}
	%\imprimirerrata{elementos-pre-textuais/errata}
	% \imprimirfolhadeaprovacao
	%\imprimirdedicatoria{1-pre-textuais/dedicatoria}
	%\imprimiragradecimentos{1-pre-textuais/agradecimentos}
	% \imprimirepigrafe{1-pre-textuais/epigrafe}
	% \imprimirabstract{1-pre-textuais/abstract}
	\imprimirresumo{1-pre-textuais/resumo}
	% \imprimirlistadeilustracoes
	% \imprimirlistadetabelas
	%\imprimirlistadequadros
	%\imprimirlistadealgoritmos
	%\imprimirlistadecodigosfonte
	% \imprimirlistadeabreviaturasesiglas
	% \imprimirlistadesimbolos{1-pre-textuais/lista-de-simbolos}   
	\imprimirsumario
	
	\setcounter{table}{0}% Deixe este comando antes da primeira tabela.
	
	%Elementos textuais
	\textual
	\chapter{Introdução}
\label{cap:introducao}

Jogos em geral, sempre tiveram um papel importante socialmente ao servir, muitas vezes como fio condutor para interações social entre familiares, amigos e até mesmo desconhecidos que por muitas vezes acabam por desenvolver laços mais tarde. Além disso, Jogos também tem mostrado ao longo do tempo seu valor competitivo, ou seja, podem ser usados para guiar competições como o CBLOL que movimentam um ecossistema econômico de torcedores, jogadores e equipes, onde muitos podem até mesmo viver exclusivamente de jogar e é capaz de lotar arenas \cite{FinalCBL40:online}. Além disso, jogos podem ser benéficos para desenvolvimento e manutenção de habilidades cognitivas como leitura \cite{Enhancingreading} ou memoria \cite{videogameimproveshigh-fidelitymemory}.

Contudo, pessoas com deficiência física muitas vezes acabam por ser excluídas desses jogos e de poderem aproveitar seus benefícios, e dentre esses grupos as pessoas com deficiência visual grave ou completamente cegas acabam por ser um dos grupos mais prejudicados e excluídos em relação aos benefícios dos jogos.

Nesse contexto, surge a necessidade do desenvolvimento de soluções e jogos capazes não so de atender as pessoas portadores de tais deficiências mas de integra-las a outras pessoas, uma vez que a capacidade integradora dos jogos é umas das perspectivas mais importantes e produtivas socialmente. 

Dessa forma, buscando alcançar esse objetivo, esse projeto propõe como o um jogo, neste caso um jogo da memoria, pode ser desenvolvido fazendo uso das tecnologias existentes no mercado afim de produzir um experiência integradora entre pessoas que possuem diferentes deficiências visuais e pessoas que não as possuem. 

\section{Perguntas de partida}

Assim, a partir dessa proposta proposta surgem as seguintes perguntas de partida, com relação ao desenvolvimento específico da solução:

\begin{questao}
  \item Quais estratégias da engenharia de software podem ser empregadas no desenvolvimento de jogos online multiplayer via navegador para garantir a qualidade do software?
  \item É possível realizar, dentro da dinâmica de um jogo em tempo real via navegador, uma experiência de uso responsiva e veloz?
  \item Existem tecnologias dentro do software livre podem conseguir fazer um jogo online em tempo real ser acessível para pessoal cegas ou com deficiência visual grave?
\end{questao}

\section{Objetivos Gerais}

O objetivo desse projeto é construir uma tecnologia que permita a criação, manutenção e distribuição de desafios de jogo da memória online multiplayer com acessibilidade total para pessoas cegas ou com deficiência visual grave. Nesse sentido, superar por meio da tecnologia as barreiras que a deficiência impõem ao integrar pessoas com deficiência com pessoas sem deficiência dentro de uma atividade recreativa de jogos online e o principal foco desse desenvolvimento.

Dessa forma, um bom uso das técnicas de desenvolvimento web, em conjunto de técnicas técnicas de engenharia de software para planejar, projetar e desenvolver o software, assim como, a construção de estruturas e nuvem serão conhecimentos essenciais para superar esse desafio.

Além disso, é importante que este desafio seja superado de forma consistente, durável e estendível. Ou seja, o software deve ser construído de modo a facilitar manutenções, adições de novas funções e bom desempenho ao longo do tempo alcançando os objetivos descritos pela engenharia de software

\section{Objetivos do software como produto}

Nessa perspectiva, surgem como objetivos finais, construir um software capaz de alcançar de maneira razoável os atributos definidos pela ISO/IEC, ou seja: \cite{ISO/IEC9126}

\begin{alineascomponto}
  \item Funcionalidade
  \item Confiabilidade
  \item Usabilidade
  \item Eficiência
  \item Manutenibilidade
  \item Portabilidade
\end{alineascomponto}



	% \input{2-textuais/2-fundamentacao-teorica}
	% \input{2-textuais/3-metodologia}
	\input{2-textuais/3-material-e-metodos.tex}
	% \input{2-textuais/4-resultados}
	% \input{2-textuais/5-conclusao}
	
	%Elementos pós-textuais	
	\bibliography{3-pos-textuais/referencias}
%	\imprimirglossario	
	% \imprimirapendices
	% 	% Adicione aqui os apendices do seu trabalho
	% 	\input{3-pos-textuais/apendices/apendice-a}
	% 	\input{3-pos-textuais/apendices/apendice-b}
	% 	\input{3-pos-textuais/apendices/apendice-c}
	% 	\input{3-pos-textuais/apendices/apendice-d}
	% \imprimiranexos
	% 	% Adicione aqui os anexos do seu trabalho
	% 	\input{3-pos-textuais/anexos/anexo-a}
	% 	\input{3-pos-textuais/anexos/anexo-b}		
	% \imprimirindice

\end{document}